\section{Focused Demographic Cluster Analysis}
\label{sec:focused_demographic_analysis}

To specifically analyze demographic patterns in generated faces, we developed a focused cluster analysis framework that targets gender, age, and ethnicity characteristics. This specialized analysis provides insights into potential biases and distribution patterns in the generated face dataset.

\subsection{Analysis Framework Design}
\label{subsec:focused_framework_design}

The focused cluster analysis framework is specifically engineered for face image analysis, with feature extraction methods tailored to demographic characteristics:

\begin{figure}[h]
\centering
\includegraphics[width=0.8\textwidth]{cluster_demographic_analysis.png}
\caption{Focused cluster analysis results showing gender and age pattern separation}
\label{fig:focused_cluster_analysis}
\end{figure}

\subsubsection{Specialized Feature Extraction}
The framework extracts face-relevant features through:

\begin{equation}
\text{Face Features} = [\text{Skin Tone}, \text{Hair Color}, \text{Face Shape}, \text{Texture}, \text{Symmetry}]
\end{equation}

where each component includes:
\begin{itemize}
    \item \textbf{Skin Tone Features}: RGB mean values from central face region, brightness, and color variation
    \item \textbf{Hair Color Features}: RGB mean from top image region, brightness
    \item \textbf{Face Shape Features}: Edge strength in facial region for jawline detection
    \item \textbf{Texture Features}: Laplacian-based texture energy for wrinkle analysis
    \item \textbf{Symmetry Features}: Horizontal symmetry score for facial symmetry assessment
\end{itemize}

\subsection{Demographic Analysis Methods}
\label{subsec:demographic_analysis_methods}

\subsubsection{Gender Pattern Analysis}
Gender estimation uses facial feature heuristics:

\begin{equation}
\text{Gender Score} = \alpha \cdot \text{Jaw Strength} + \beta \cdot \text{Brow Contrast}
\end{equation}

where:
\begin{itemize}
    \item $\text{Jaw Strength} = \text{Mean}(\text{Sobel}(\text{Bottom Region}))$ - indicates jawline definition
    \item $\text{Brow Contrast} = \sigma(\text{Brow Region})$ - indicates brow ridge prominence
    \item $\alpha = 0.6, \beta = 0.4$ - empirical weights
\end{itemize}

\subsubsection{Age Pattern Analysis}
Age estimation combines multiple texture and contrast features:

\begin{equation}
\text{Age Score} = \gamma \cdot \text{Texture Energy} + \delta \cdot \text{Contrast} - \epsilon \cdot \text{Smoothness}
\end{equation}

where:
\begin{itemize}
    \item $\text{Texture Energy} = \text{Mean}(|\nabla^2 G_{\sigma=1}(I)|)$ - Laplacian of Gaussian for wrinkle detection
    \item $\text{Contrast} = \sigma(I)$ - overall image contrast
    \item $\text{Smoothness} = \text{Mean}(|I - G_{\sigma=2}(I)|)$ - deviation from smoothed version
    \item $\gamma = 0.5, \delta = 0.3, \epsilon = 0.2$ - empirical weights
\end{itemize}

\subsubsection{Ethnicity Pattern Analysis}
Skin tone analysis uses HSV color space:

\begin{equation}
\text{Ethnicity Category} = f(H_{\text{skin}}, S_{\text{skin}}, V_{\text{skin}})
\end{equation}

where conversion from RGB to HSV is performed using:
\begin{equation}
\begin{aligned}
V &= \max(R,G,B) \\
S &= \begin{cases}
\frac{V - \min(R,G,B)}{V} & \text{if } V > 0 \\
0 & \text{otherwise}
\end{cases} \\
H &= \begin{cases}
0 & \text{if } V = \min(R,G,B) \\
60 \times \frac{G - B}{V - \min(R,G,B)} & \text{if } V = R \\
60 \times \left(2 + \frac{B - R}{V - \min(R,G,B)}\right) & \text{if } V = G \\
60 \times \left(4 + \frac{R - G}{V - \min(R,G,B)}\right) & \text{if } V = B
\end{cases}
\end{aligned}
\end{equation}

\subsection{Analysis Results}
\label{subsec:focused_results}

\subsubsection{Cluster Distribution}
Analysis of 64 generated faces revealed two main clusters:

\begin{table}[h]
\centering
\begin{tabular}{|l|c|c|}
\hline
\textbf{Cluster} & \textbf{Number of Images} & \textbf{Percentage} \\
\hline
Cluster 0 & 43 & 67.2\% \\
Cluster 1 & 21 & 32.8\% \\
\hline
\end{tabular}
\caption{Distribution of generated faces across identified clusters}
\label{tab:focused_cluster_distribution}
\end{table}

\subsubsection{Gender Analysis Results}
Both clusters showed characteristics potentially indicative of male faces:

\begin{table}[h]
\centering
\begin{tabular}{|l|c|c|c|}
\hline
\textbf{Cluster} & \textbf{Jaw Strength} & \textbf{Brow Contrast} & \textbf{Gender Estimation} \\
\hline
Cluster 0 & 0.325 & 0.356 & Potentially male \\
Cluster 1 & 0.259 & 0.322 & Potentially male \\
\hline
\end{tabular}
\caption{Gender pattern analysis results}
\label{tab:gender_analysis}
\end{table}

The higher jaw strength and brow contrast values in both clusters suggest predominantly masculine facial features in the generated dataset.

\subsubsection{Age Analysis Results}
Both clusters displayed characteristics potentially indicating older individuals:

\begin{table}[h]
\centering
\begin{tabular}{|l|c|c|c|c|}
\hline
\textbf{Cluster} & \textbf{Texture Energy} & \textbf{Smoothness} & \textbf{Contrast} & \textbf{Age Estimation} \\
\hline
Cluster 0 & 0.043 & 0.068 & 0.354 & Potentially older \\
Cluster 1 & 0.031 & 0.049 & 0.285 & Potentially older \\
\hline
\end{tabular}
\caption{Age pattern analysis results}
\label{tab:age_analysis}
\end{table}

Higher texture energy and contrast values in Cluster 0 suggest more pronounced aging characteristics compared to Cluster 1.

\subsubsection{Ethnicity Analysis Results}
Skin tone analysis revealed distinct color characteristics:

\begin{table}[h]
\centering
\begin{tabular}{|l|c|c|c|c|}
\hline
\textbf{Cluster} & \textbf{Skin Tone (RGB)} & \textbf{Brightness} & \textbf{Saturation} & \textbf{Ethnicity Estimation} \\
\hline
Cluster 0 & (0.69, 0.61, 0.38) & 0.69 & 0.46 & Mixed or unclear \\
Cluster 1 & (0.53, 0.42, 0.16) & 0.53 & 0.69 & Potentially medium/warm \\
\hline
\end{tabular}
\caption{Ethnicity pattern analysis results}
\label{tab:ethnicity_analysis}
\end{table}

Cluster 0 shows higher brightness and lower saturation, while Cluster 1 exhibits warmer, more saturated skin tones.

\subsection{Methodological Details}
\label{subsec:focused_methodology}

\subsubsection{Facial Region Detection}
The analysis employs a simplified facial region detection heuristic:

\begin{verbatim}
# Extract center region where face is typically located
height, width = image.shape[:2]
center_h, center_w = height // 2, width // 2
crop_size = min(height, width) // 3  # 1/3 of smaller dimension

face_region = image[center_h-crop_size:center_h+crop_size,
                    center_w-crop_size:center_w+crop_size]
\end{verbatim}

\subsubsection{Texture Analysis Implementation}
Age-related texture analysis uses Gaussian Laplacian for wrinkle detection:

\begin{verbatim}
# Compute texture energy using Laplacian of Gaussian
from scipy.ndimage import gaussian_laplace
laplacian = gaussian_laplace(grayscale_image, sigma=1.0)
texture_energy = np.mean(np.abs(laplacian))
\end{verbatim}

\subsubsection{Symmetry Calculation}
Facial symmetry is computed using horizontal reflection:

\begin{verbatim}
# Calculate horizontal symmetry score
left_half = image[:, :width//2]
right_half = np.fliplr(image[:, width//2:])
symmetry_score = np.mean(np.abs(left_half - right_half))
\end{verbatim}

\subsection{Limitations and Considerations}
\label{subsec:focused_limitations}

\subsubsection{Methodological Limitations}
\begin{enumerate}
    \item \textbf{Simplified feature extraction}: Relies on geometric heuristics rather than learned facial landmarks
    \item \textbf{Binary gender assumptions}: Uses traditional male/female facial feature stereotypes
    \item \textbf{Ethnicity oversimplification}: Reduces complex ethnic characteristics to skin tone metrics
    \item \textbf{No ground truth validation}: Cannot verify accuracy without labeled dataset
\end{enumerate}

\subsubsection{Ethical Considerations}
\begin{itemize}
    \item Gender and ethnicity classifications are algorithmic estimates, not definitive labels
    \item Analysis may perpetuate stereotypes if interpreted uncritically
    \item Results should be used for understanding model behavior, not for making demographic claims
    \item Manual validation is essential for any applied use cases
\end{itemize}

\subsection{Practical Applications}
\label{subsec:focused_applications}

\subsubsection{Bias Detection and Mitigation}
The analysis can help identify:
\begin{itemize}
    \item Gender imbalance in generated faces
    \item Age distribution biases
    \item Ethnic diversity or lack thereof
    \item Systematic patterns in facial feature generation
\end{itemize}

\subsubsection{Dataset Curation Guidance}
Results can inform:
\begin{itemize}
    \item Targeted generation strategies to improve diversity
    \item Dataset balancing efforts
    \item Model fine-tuning for specific demographic groups
    \begin{verbatim}
# Example: Generating more diverse age range
if cluster_analysis['age_estimation'] == 'Potentially older':
    adjust_generation_parameters(reduce_texture=True)
    \end{verbatim}
\end{itemize}

\subsection{Visualization and Reporting}
\label{subsec:focused_visualization}

The framework generates comprehensive outputs:

\subsubsection{Cluster Collages}
Each cluster is visualized as a collage grid showing representative images, allowing for manual pattern recognition and validation of algorithmic classifications.

\subsubsection{Demographic Analysis Plot}
The main visualization (Figure \ref{fig:focused_cluster_analysis}) shows:
\begin{enumerate}
    \item PCA projection of face features colored by cluster assignment
    \item Bar chart of cluster sizes
    \item Gender pattern scatter plot with heuristic classification regions
    \item Age pattern scatter plot with heuristic classification regions
\end{enumerate}

\subsubsection{Comprehensive Report}
The automated report includes:
\begin{itemize}
    \item Dataset summary and cluster distribution
    \item Detailed gender, age, and ethnicity analysis
    \item Feature metrics and algorithmic estimations
    \item Recommendations for further analysis
    \item Paths to all generated visualizations and data files
\end{itemize}

\subsection{Conclusions and Future Directions}
\label{subsec:focused_conclusions}

The focused cluster analysis revealed that:
\begin{enumerate}
    \item Generated faces cluster into two main groups with 67\%-33\% distribution
    \item Both clusters show predominantly masculine facial characteristics
    \item Age characteristics suggest older individuals in both clusters
    \item Skin tone variations suggest some ethnic diversity but limited range
\end{enumerate}

\subsubsection{Recommended Improvements}
Future work should address:
\begin{itemize}
    \item Integration of pre-trained demographic classifiers
    \item More sophisticated facial landmark detection
    \item Multi-dimensional demographic representation
    \item Ground truth validation with labeled datasets
    \item Intersectional analysis considering multiple demographic dimensions simultaneously
\end{itemize}

This focused analysis framework provides a valuable starting point for understanding demographic patterns in generated faces, though results should be interpreted with awareness of methodological limitations and potential biases.
