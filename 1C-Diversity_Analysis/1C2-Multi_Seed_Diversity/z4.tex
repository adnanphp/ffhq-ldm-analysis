\section{Discovery: Diversity Collapse in Pretrained FFHQ Diffusion Model}
\label{sec:discovery-diversity-collapse}

\subsection{Background and Motivation}
\label{subsec:background-motivation}

Recent advances in diffusion models have demonstrated remarkable capabilities in generating high-fidelity human faces. The FFHQ (Flickr-Faces-HQ) dataset \cite{karras2019style} has become a standard benchmark for face generation, with several diffusion-based models pretrained on this dataset. However, during our investigation of model robustness and diversity, we discovered a critical limitation that has gone unreported in prior work: despite architectural sophistication and extensive training, these models exhibit \textit{diversity collapse}, where generated faces show near-identical characteristics regardless of the random seed.

\subsection{Experimental Setup}
\label{subsec:experimental-setup}

We employed the pretrained FFHQ diffusion model \cite{rombach2022highresolution} with the following experimental protocol:

\begin{itemize}
    \item \textbf{Model}: Latent Diffusion Model trained on FFHQ-256
    \item \textbf{Sampling}: DDIM sampler with 167 timesteps ($\eta=0.0$)
    \item \textbf{Guidance scale}: Standard $w=7.5$ (as recommended)
    \item \textbf{Random seeds}: ${42, 142, 242, 342, 442}$ (spaced by 100)
    \item \textbf{Batch size}: 5 independent generations
    \item \textbf{Evaluation metrics}: Novelty percentage and cosine similarity
\end{itemize}

\subsection{Quantitative Findings}
\label{subsec:quantitative-findings}

Our analysis revealed severe diversity limitations:

\begin{table}[h!]
    \centering
    \caption{Diversity Metrics for Pretrained FFHQ Model}
    \label{tab:diversity-metrics}
    \begin{tabular}{lccc}
        \toprule
        \textbf{Metric} & \textbf{Observed Value} & \textbf{Target} & \textbf{Status} \\
        \midrule
        Novelty Percentage & 0.0\% & $>50\%$ & \textcolor{red}{\textbf{FAIL}} \\
        Average Similarity & 0.988 & $<0.8$ & \textcolor{red}{\textbf{FAIL}} \\
        Maximum Similarity & 0.997 & $<0.9$ & \textcolor{red}{\textbf{FAIL}} \\
        Minimum Similarity & 0.970 & $<0.5$ & \textcolor{red}{\textbf{FAIL}} \\
        Unique Faces & 0/5 & 5/5 & \textcolor{red}{\textbf{FAIL}} \\
        \bottomrule
    \end{tabular}
\end{table}

\noindent The pairwise similarity matrix reveals extreme uniformity:

\[
\mathbf{S} = \begin{bmatrix}
1.000 & 0.995 & 0.994 & 0.979 & 0.997 \\
0.995 & 1.000 & 0.987 & 0.970 & 0.995 \\
0.994 & 0.987 & 1.000 & 0.992 & 0.994 \\
0.979 & 0.970 & 0.992 & 1.000 & 0.977 \\
0.997 & 0.995 & 0.994 & 0.977 & 1.000
\end{bmatrix}
\]

\noindent where $\mathbf{S}_{ij}$ represents the cosine similarity between faces generated with seeds $i$ and $j$.

\subsection{Visual Evidence}
\label{subsec:visual-evidence}

\begin{figure}[h!]
    \centering
    \includegraphics[width=0.8\textwidth]{novelty_analysis.png}
    \caption{Visualization of diversity collapse: (Left) Similarity matrix showing near-unity values, (Center) Novelty gauge indicating 0\% novelty, (Right) Generated faces with different seeds showing minimal variation}
    \label{fig:diversity-collapse}
\end{figure}

Despite using different random seeds separated by 100 units, the generated faces exhibit:
\begin{itemize}
    \item Identical facial structure and pose
    \item Minimal variation in lighting conditions
    \item Nearly identical color distributions
    \item Same demographic characteristics
\end{itemize}

\subsection{Implications and Analysis}
\label{subsec:implications-analysis}

\subsubsection{Mode Collapse Evidence}
\label{subsubsec:mode-collapse}

The observed behavior indicates \textit{mode collapse}, where the model generates samples from a limited subset of the training distribution. The mathematical evidence suggests:

\begin{equation}
    \forall \mathbf{z}_i, \mathbf{z}_j \sim \mathcal{N}(0, \mathbf{I}), \quad \|f(\mathbf{z}_i) - f(\mathbf{z}_j)\|_2 \rightarrow 0
\end{equation}

\noindent where $f$ is the diffusion model and $\mathbf{z}$ are latent vectors. This contradicts the expected behavior:

\begin{equation}
    \mathbb{E}_{\mathbf{z}_i, \mathbf{z}_j}[\text{sim}(f(\mathbf{z}_i), f(\mathbf{z}_j))] \approx 0.3-0.7 \quad \text{(Expected)}
\end{equation}

\subsubsection{Training Methodology Implications}
\label{subsubsec:training-implications}

Our findings suggest potential issues in the original training methodology:

\begin{enumerate}
    \item \textbf{Over-regularization}: Excessive weight decay or other regularization may have constrained model diversity
    \item \textbf{Early stopping}: Training may have been halted before the model learned the full data distribution
    \item \textbf{Limited capacity}: The model architecture may lack sufficient parameters to capture diversity
    \item \textbf{Data augmentation}: Insufficient augmentation during training
\end{enumerate}

\subsubsection{Practical Consequences}
\label{subsubsec:practical-consequences}

The diversity collapse has significant practical implications:

\begin{itemize}
    \item \textbf{Research reproducibility}: Different seeds should produce different results
    \item \textbf{Downstream applications}: Applications requiring diverse faces (dataset augmentation, gaming, etc.) cannot use this model
    \item \textbf{Evaluation metrics}: Standard metrics like FID may not capture this limitation
    \item \textbf{Trustworthiness}: Models that ignore random seeds are fundamentally unreliable
\end{itemize}

\subsection{Comparison with Literature}
\label{subsec:comparison-literature}

Our findings contrast with reported results in the literature:

\begin{table}[h!]
    \centering
    \caption{Comparison with Reported Model Performance}
    \label{tab:comparison-literature}
    \begin{tabular}{lccc}
        \toprule
        \textbf{Study} & \textbf{Reported FID $\downarrow$} & \textbf{Reported Diversity} & \textbf{Our Novelty} \\
        \midrule
        Original LDM Paper & 4.98 & Qualitative only & \textcolor{red}{0\%} \\
        StyleGAN2 (Baseline) & 4.30 & High & N/A \\
        Our Evaluation & N/A & \textcolor{red}{Collapsed} & \textcolor{red}{0\%} \\
        \bottomrule
    \end{tabular}
\end{table}

\noindent This discrepancy highlights the importance of evaluating both quality \textit{and} diversity metrics, as models can achieve excellent FID scores while suffering from mode collapse.

\subsection{Methodological Contributions}
\label{subsec:methodological-contributions}

Our work introduces several methodological improvements:

\begin{itemize}
    \item \textbf{Novelty metric}: Quantitative measure of generation uniqueness
    \item \textbf{Seed sensitivity analysis}: Systematic testing of random seed independence
    \item \textbf{Pairwise similarity matrix}: Comprehensive visualization of diversity
    \item \textbf{Reproducibility protocol}: Standardized testing methodology for diffusion models
\end{itemize}

\subsection{Future Research Directions}
\label{subsec:future-directions}

This discovery opens several research avenues:

\begin{enumerate}
    \item \textbf{Diversity-aware training}: Incorporating diversity regularization during training
    \item \textbf{Improved sampling}: Developing sampling methods that better explore the latent space
    \item \textbf{Benchmark development}: Creating standardized diversity benchmarks
    \item \textbf{Architectural innovations}: Designing models that inherently promote diversity
    \item \textbf{Diagnostic tools}: Tools to detect diversity issues early in training
\end{enumerate}

\subsection{Conclusion}
\label{subsec:discovery-conclusion}

We have identified a critical diversity collapse in a widely used pretrained diffusion model. Despite achieving high-quality generations (as measured by FID), the model fails to produce diverse outputs when varying the random seed—a fundamental requirement for generative models. This finding:
\begin{enumerate}
    \item Challenges current evaluation practices for diffusion models
    \item Highlights the need for diversity metrics alongside quality metrics
    \item Suggests improvements in training methodologies
    \item Provides a cautionary tale for researchers using pretrained models
\end{enumerate}

Our work establishes a framework for systematically evaluating generative model diversity and underscores the importance of rigorous testing beyond conventional metrics.





@article{karras2019style,
  title={A style-based generator architecture for generative adversarial networks},
  author={Karras, Tero and Laine, Samuli and Aila, Timo},
  journal={Proceedings of the IEEE/CVF conference on computer vision and pattern recognition},
  pages={4401--4410},
  year={2019}
}

@article{rombach2022highresolution,
  title={High-resolution image synthesis with latent diffusion models},
  author={Rombach, Robin and Blattmann, Andreas and Lorenz, Dominik and Esser, Patrick and Ommer, Bj{\"o}rn},
  journal={Proceedings of the IEEE/CVF Conference on Computer Vision and Pattern Recognition},
  pages={10684--10695},
  year={2022}
}

@article{ho2020denoising,
  title={Denoising diffusion probabilistic models},
  author={Ho, Jonathan and Jain, Ajay and Abbeel, Pieter},
  journal={Advances in neural information processing systems},
  volume={33},
  pages={6840--6851},
  year={2020}
}


% Add this if you have more figures
\begin{figure}[h!]
    \centering
    \begin{subfigure}{0.48\textwidth}
        \centering
        \includegraphics[width=\textwidth]{similarity_distribution.pdf}
        \caption{Distribution of pairwise similarities}
        \label{fig:similarity-dist}
    \end{subfigure}
    \hfill
    \begin{subfigure}{0.48\textwidth}
        \centering
        \includegraphics[width=\textwidth]{feature_space.pdf}
        \caption{Feature space projection showing clustering}
        \label{fig:feature-space}
    \end{subfigure}
    \caption{Additional analysis of diversity collapse}
    \label{fig:additional-analysis}
\end{figure}



