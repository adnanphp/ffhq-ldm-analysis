\section{Detailed Cluster Analysis}
\label{sec:detailed_cluster_analysis}

To understand the underlying structure and identify natural groupings within the generated faces, we performed comprehensive cluster analysis. This analysis aims to discover meaningful patterns such as gender, age, ethnicity, or visual characteristics that emerge from the generated data.

\subsection{Analysis Framework}
\label{subsec:analysis_framework}

The cluster analysis framework provides a systematic approach to identifying and characterizing natural groupings within generated face images. The implementation includes multiple clustering algorithms, comprehensive feature extraction, and detailed visualization capabilities.

\subsubsection{Architecture Overview}
\begin{itemize}
    \item \textbf{Input Processing}: Loads and processes PNG images from generated face datasets
    \item \textbf{Feature Extraction}: Extracts multiple types of features per image
    \item \textbf{Optimal Cluster Determination}: Uses multiple metrics to determine ideal cluster count
    \item \textbf{Multi-Algorithm Clustering}: Applies KMeans, DBSCAN, Agglomerative, and Gaussian Mixture Models
    \item \textbf{Characterization \& Visualization}: Analyzes cluster characteristics and creates comprehensive visualizations
\end{itemize}

\subsubsection{Feature Extraction Pipeline}
The framework extracts 28-dimensional feature vectors from each image, including:

\begin{equation}
\text{Feature Vector} = [\text{Color Features}, \text{Texture Features}, \text{Shape Features}, \text{Brightness Features}, \text{Histogram Moments}]
\end{equation}

Where:
\begin{itemize}
    \item \textbf{Color Features}: Mean, standard deviation, 25th and 75th percentiles for each RGB channel (12 features)
    \item \textbf{Texture Features}: Gradient magnitude statistics using Sobel operators (4 features)
    \item \textbf{Shape Features}: Edge density using Gaussian gradient magnitude (1 feature)
    \item \textbf{Brightness Features}: Average brightness and contrast (2 features)
    \item \textbf{Histogram Moments}: Normalized histogram statistics for each channel (9 features)
\end{itemize}

\subsection{Optimal Cluster Determination}
\label{subsec:optimal_cluster_determination}

To determine the optimal number of clusters, we employed multiple statistical methods:

\subsubsection{Silhouette Analysis}
\begin{equation}
s(i) = \frac{b(i) - a(i)}{\max\{a(i), b(i)\}}
\end{equation}

where $a(i)$ is the average distance between sample $i$ and other points in the same cluster, and $b(i)$ is the minimum average distance from sample $i$ to points in a different cluster.

\subsubsection{Elbow Method}
Identifies the "elbow point" where increasing clusters provides diminishing returns in reducing within-cluster sum of squares (inertia):

\begin{equation}
\text{Inertia} = \sum_{i=1}^{n} \min_{\mu_j \in C} (||x_i - \mu_j||^2)
\end{equation}

\subsubsection{Bayesian Information Criterion (BIC)}
\begin{equation}
\text{BIC} = \ln(n)k - 2\ln(\hat{L})
\end{equation}
where $n$ is the number of samples, $k$ is the number of parameters, and $\hat{L}$ is the maximized value of the likelihood function.

\subsubsection{Akaike Information Criterion (AIC)}
\begin{equation}
\text{AIC} = 2k - 2\ln(\hat{L})
\end{equation}

\subsubsection{Optimal Cluster Results}
Based on our analysis of 64 generated faces:

\begin{table}[h]
\centering
\begin{tabular}{|l|c|}
\hline
\textbf{Method} & \textbf{Recommended Clusters} \\
\hline
Silhouette Score & 2 \\
Elbow Method & 4 \\
Bayesian Information Criterion (BIC) & 6 \\
Akaike Information Criterion (AIC) & 7 \\
\hline
\rowcolor{gray!20} \textbf{Final Recommendation} & \textbf{2} \\
\hline
\end{tabular}
\caption{Optimal cluster recommendations by different methods}
\label{tab:optimal_clusters}
\end{table}

\subsection{Clustering Algorithm Performance}
\label{subsec:clustering_performance}

We evaluated four clustering algorithms with the recommended 2-cluster configuration:

\begin{table}[h]
\centering
\begin{tabular}{|l|c|c|c|}
\hline
\textbf{Algorithm} & \textbf{Silhouette Score} & \textbf{Calinski-Harabasz} & \textbf{Davies-Bouldin} \\
\hline
KMeans & 0.293 & 31.1 & 1.206 \\
Agglomerative & 0.281 & 30.9 & 1.277 \\
Gaussian Mixture Model & 0.293 & 31.1 & 1.206 \\
\hline
\end{tabular}
\caption{Performance metrics for different clustering algorithms}
\label{tab:clustering_performance}
\end{table}

All algorithms showed consistent performance, with KMeans and GMM performing slightly better than Agglomerative clustering. The moderate silhouette scores (around 0.29) suggest that clusters have some separation but not perfectly distinct boundaries.

\subsection{Cluster Characteristics}
\label{subsec:cluster_characteristics}

\subsubsection{Cluster Distribution}
\begin{table}[h]
\centering
\begin{tabular}{|l|c|c|}
\hline
\textbf{Cluster} & \textbf{Number of Images} & \textbf{Percentage} \\
\hline
Cluster 0 & 43 & 67.2\% \\
Cluster 1 & 21 & 32.8\% \\
\hline
\textbf{Total} & \textbf{64} & \textbf{100\%} \\
\hline
\end{tabular}
\caption{Distribution of images across clusters}
\label{tab:cluster_distribution}
\end{table}

\subsubsection{Feature Analysis}
Both clusters showed similar characteristics:
\begin{itemize}
    \item \textbf{Dark}: Low average brightness levels
    \item \textbf{Colorful}: High color variation within images
    \item \textbf{Textured}: High gradient magnitudes indicating detailed features
\end{itemize}

The similarity in characteristics suggests that while the clustering algorithm successfully separates images into two groups, the visual distinction may be subtle rather than categorical.

\subsection{Visualization Methods}
\label{subsec:visualization_methods}

\subsubsection{Cluster Metrics Comparison}
Figure \ref{fig:cluster_metrics} shows the comprehensive comparison of different clustering metrics across varying numbers of clusters. This visualization helps identify trade-offs between cluster compactness and separation.

\subsubsection{Algorithm Comparison}
Figure \ref{fig:clustering_algorithms} compares the results of different clustering algorithms in a 2D PCA projection space, showing how each algorithm partitions the feature space.

\subsubsection{Detailed KMeans Analysis}
Figure \ref{fig:kmeans_detailed} provides:
\begin{enumerate}
    \item Cluster visualization with centroids and Voronoi regions
    \item Bar chart showing cluster sizes
    \item Silhouette plot for each cluster
\end{enumerate}

\subsubsection{Cluster Collages}
\begin{figure}[h]
\centering
\includegraphics[width=0.8\textwidth]{cluster_collage_example.png}
\caption{Example cluster collage showing 16 sample images from Cluster 0}
\label{fig:cluster_collage}
\end{figure}

For each cluster, we created visual collages showing all images within that cluster, allowing for manual inspection of common characteristics.

\subsection{Interpretation and Insights}
\label{subsec:cluster_interpretation}

\subsubsection{Silhouette Score Analysis}
The silhouette score of 0.293 indicates:
\begin{itemize}
    \item Samples are reasonably well-clustered
    \item Some overlap exists between clusters
    \item Clusters have internal cohesion but moderate separation
\end{itemize}

\subsubsection{Potential Categorical Basis}
Based on cluster characteristics and visual inspection, the clusters may represent:
\begin{itemize}
    \item \textbf{Gender distinction}: Male vs. female faces
    \item \textbf{Age grouping}: Young vs. older faces
    \item \textbf{Ethnicity patterns}: Different ethnic characteristics
    \item \textbf{Lighting conditions}: Variations in image brightness and contrast
\end{itemize}

\subsubsection{DBSCAN Results}
DBSCAN (Density-Based Spatial Clustering of Applications with Noise) was also applied but found limited utility with the current feature representation, suggesting relatively uniform density in the feature space.

\subsection{Limitations and Future Work}
\label{subsec:cluster_limitations}

\subsubsection{Current Limitations}
\begin{enumerate}
    \item \textbf{Feature representation}: Current features focus on low-level visual properties rather than semantic content
    \item \textbf{Cluster granularity}: Only two clusters identified, potentially oversimplifying the data structure
    \item \textbf{Semantic gap}: Visual features may not correspond directly to human-perceptible categories
\end{enumerate}

\subsubsection{Recommended Improvements}
\begin{enumerate}
    \item \textbf{Semantic feature extraction}: Use pre-trained face recognition models (FaceNet, ArcFace) for identity-based features
    \item \textbf{Demographic attribute prediction}: Integrate age, gender, and ethnicity classifiers
    \item \textbf{Hierarchical clustering}: Explore multi-level clustering for better granularity
    \item \textbf{Interactive exploration}: Develop tools for manual cluster refinement and labeling
\end{enumerate}

\subsection{Technical Implementation Details}
\label{subsec:technical_implementation}

\subsubsection{Software Architecture}
The analysis framework is implemented with the following key components:

\begin{verbatim}
DetailedClusterAnalyzer
├── load_images_and_features()    # Data loading and preprocessing
├── extract_image_features()      # Multi-dimensional feature extraction
├── determine_optimal_clusters()  # Statistical cluster count determination
├── perform_clustering()          # Multi-algorithm clustering
├── analyze_cluster_characteristics() # Cluster description
└── generate_comprehensive_report()  # Result compilation
\end{verbatim}

\subsubsection{Computational Requirements}
\begin{itemize}
    \item \textbf{Input}: 64 PNG images (256×256 resolution)
    \item \textbf{Feature matrix}: 64×28 dimensions
    \item \textbf{Runtime}: Approximately 2-3 minutes on standard hardware
    \item \textbf{Memory}: < 500MB for all computations
\end{itemize}

\subsection{Conclusions from Cluster Analysis}
\label{subsec:cluster_conclusions}

The cluster analysis revealed that:
\begin{enumerate}
    \item The generated faces naturally group into two main clusters
    \item Cluster sizes are imbalanced (67\% vs. 33\%), suggesting potential systematic bias in generation
    \item Both clusters share similar low-level visual characteristics (dark, colorful, textured)
    \item Different clustering algorithms produce consistent results
    \item Further investigation with semantic features is needed to interpret cluster meanings
\end{enumerate}

The framework provides a solid foundation for understanding the structure of generated face datasets and can be extended with more sophisticated feature extraction methods to uncover deeper semantic patterns.
