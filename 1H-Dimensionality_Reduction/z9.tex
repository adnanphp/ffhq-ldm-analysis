\section{PCA Analysis of Generated Images}

\subsection{Report Summary}
A principal component analysis (PCA) was performed on 64 generated images to examine their structure in pixel space. The analysis revealed the following key results:

\begin{itemize}
    \item \textbf{Images analyzed:} 64
    \item \textbf{Original feature dimension:} 196,608 (flattened pixels)
    \item \textbf{Memory footprint:} 96.0 MB
    \item \textbf{Total pixels analyzed:} 12,582,912
    \item \textbf{Reduced dimension:} 50 principal components
    \item \textbf{Total variance explained:} 93.852\%
    \item \textbf{Components for 95\% variance:} 1
    \item \textbf{Components for 99\% variance:} 1
\end{itemize}

\subsection{Variance Explained by Principal Components}
The analysis shows a rapid drop in explained variance after the first principal component, which is typical for image data:

\begin{itemize}
    \item \textbf{First PC:} 15.14\% of variance
    \item \textbf{Second PC:} 8.34\% of variance
    \item \textbf{Top 5 PCs:} 38.36\% of variance
    \item \textbf{Top 10 PCs:} 52.44\% of variance
\end{itemize}

\subsection{Interpretation}
The high variance explained by the first few principal components suggests a strong underlying structure in the generated images. The fact that only 1 component is needed to capture 95\% of the variance indicates that the images occupy a relatively low-dimensional subspace within the high-dimensional pixel space (196,608 dimensions).

This result has important implications:
\begin{itemize}
    \item The generated images exhibit high redundancy in pixel space
    \item Most visual information can be captured in a much lower-dimensional representation
    \item The generative process produces images with consistent structural patterns
\end{itemize}

\subsection{Visualizations Generated}
The analysis produced several visualization files:

\begin{table}[h]
\centering
\begin{tabular}{ll}
\hline
\textbf{Filename} & \textbf{Description} \\
\hline
\texttt{pca\_analysis.png} & Comprehensive PCA visualization (6 subplots) \\
\texttt{pca\_variance\_detailed.png} & Detailed variance plots \\
\texttt{pca\_scatter\_with\_labels.png} & PC1 vs PC2 with sample labels \\
\texttt{pca\_biplot.png} & Biplot showing samples and principal directions \\
\hline
\end{tabular}
\caption{PCA visualization files generated}
\end{table}

\subsection{Data Files}
The analysis produced the following data files:
\begin{itemize}
    \item \texttt{pca\_transformed.npy} -- PCA-transformed data in NumPy format
    \item \texttt{pca\_results.json} -- PCA analysis results in JSON format
\end{itemize}

\subsection{Code Implementation}
The PCA analysis was performed using a custom Python framework (\texttt{latent\_space\_analysis\_simple.py}) with the following key components:

\subsubsection{Data Loading}
\begin{itemize}
    \item Loads PNG images from a specified directory
    \item Normalizes pixel values to [0, 1] range
    \item Flattens images to create feature vectors
    \item Handles memory efficiently for large datasets
\end{itemize}

\subsubsection{PCA Analysis}
\begin{itemize}
    \item Standardizes data using \texttt{StandardScaler}
    \item Performs PCA with configurable number of components (default: 50)
    \item Calculates explained variance and cumulative variance
    \item Identifies components needed for specific variance thresholds
\end{itemize}

\subsubsection{Visualization}
The code generates multiple visualization types:
\begin{enumerate}
    \item Scree plot showing explained variance by component
    \item Cumulative variance plot with 95\% and 99\% thresholds
    \item 2D and 3D scatter plots of principal components
    \item Heatmap of PCA components
    \item Distribution analysis along principal components
    \item Biplots showing sample relationships
\end{enumerate}

\subsubsection{Reporting}
\begin{itemize}
    \item Generates comprehensive text reports
    \item Saves results in JSON format for further analysis
    \item Creates organized output directory structure with timestamps
    \item Provides runtime statistics and performance metrics
\end{itemize}

\subsection{Technical Details}
\begin{itemize}
    \item \textbf{Framework:} Python with scikit-learn PCA implementation
    \item \textbf{Standardization:} Yes (z-score normalization)
    \item \textbf{Random state:} Fixed (seed = 42) for reproducibility
    \item \textbf{Output organization:} Timestamp-based directories with subfolders for visualizations and analysis data
    \item \textbf{Progress tracking:} Uses tqdm for loading progress
\end{itemize}

\subsection{Conclusions}
The PCA analysis successfully demonstrated that the generated FFHQ images occupy a low-dimensional subspace within the high-dimensional pixel space. The ability to capture 95\% of the variance with just 1 principal component indicates significant structure and redundancy in the generated images. This finding validates the effectiveness of the generative process and provides insights for potential dimensionality reduction in downstream applications.

The analysis framework provides a comprehensive toolset for examining the structure of generated images, with capabilities for visualization, reporting, and further statistical analysis.
